firmware of the dedicated motion processor.

\subsection*{Goals}

The motion processor is required to \+:
\begin{DoxyItemize}
\item keep track of the current absolute position and heading of the robot
\item provide an high level interface to accurately control the motion of the robot
\end{DoxyItemize}

The robot should be able to go to any goal position and heading using a trajectory composed of two arcs and a straight line. Here is an example \+:



Because the robot is not capable of infinite linear nor angular acceleration, maximum accelerations should be defined and the speed profiles should look like (for example) \+:



Given the intricate relation between motion and high level considerations, the computation of the trajectory won\textquotesingle{}t be performed by the motion coprocessor. It will receive position control commands \+:
\begin{DoxyItemize}
\item maximum linear and angular accelerations
\item linear and angular cruise speeds
\item goal mean distance (mean distance being the mean between the distance of the two wheels)
\item goal heading, and eventually a reference mean distance to start the rotation. This should allow for a precise synchronization between translation and rotation.
\end{DoxyItemize}

This way of controlling motion provide a flexible interface without overloading the I2C bus through which the commands are transmitted, nor requiring any real time capabilities from the master.

The Motion\+Controller will provide a {\bfseries position control}, so that the robot is either moving toward its goal or holding its current position.

The real position should also be tracked, using a fusion of the data from the encoder wheels, the I\+MU and radio measurements. A special attention should be given to the I\+MU response time. Indeed when rotating, it is suspected that the heading measurement lags.

\subsection*{Interface}

The Motion\+Controller should provide read/write data to the I2C master mapped as 8 or 16 bits registers \+:


\begin{DoxyItemize}
\item current x absolute position (read)
\item current y absolute position (read)
\item current heading (read/write)
\end{DoxyItemize}

~\newline



\begin{DoxyItemize}
\item current right wheel distance (read/write)
\item current left wheel distance (read/write)
\end{DoxyItemize}

~\newline



\begin{DoxyItemize}
\item maximum accelaration (read/write)
\item maximum angular accelaration (read/write)
\item cruise speed (read/write)
\item cruise angular speed (read/write)
\item goal mean distance (read/write)
\item goal heading (read/write)
\item heading distance sync reference (read/write)~\newline

\end{DoxyItemize}

~\newline



\begin{DoxyItemize}
\item linear P\+ID coefficients (read/write, flash stored)
\item angular P\+ID coefficients (read/write, flash stored)
\end{DoxyItemize}

Any distance is in mm. Angle range is \mbox{[}0, 360\mbox{]}.

\subsection*{\char`\"{}\+Registers\char`\"{} address and size}

The motorboard listens on address 0x12 (device address).

Configuration values are placed first, then data and finally targets. All write-\/only values are read-\/as-\/zero (R\+AZ). Writing to a read-\/only value is implementation-\/defined, it will defined later. For 32 bits value, split into 2 16-\/bit register, the L\+OW register must always be read/written first. Failing to follow this rule will lead to invalid data.

\tabulinesep=1mm
\begin{longtabu} spread 0pt [c]{*{4}{|X[-1]}|}
\hline
\rowcolor{\tableheadbgcolor}{\bf Name}&{\bf Address}&{\bf Access}&{\bf Size (in bits)  }\\\cline{1-4}
\endfirsthead
\hline
\endfoot
\hline
\rowcolor{\tableheadbgcolor}{\bf Name}&{\bf Address}&{\bf Access}&{\bf Size (in bits)  }\\\cline{1-4}
\endhead
wheels\+\_\+gap (in mm)&0x00&R/W&16 \\\cline{1-4}
ticks\+\_\+per\+\_\+m&0x02&R/W&16 \\\cline{1-4}
angular\+\_\+trust\+\_\+threshold \href{in °.s-1}{\tt absolute value}&0x04&R/W&16 \\\cline{1-4}
max\+\_\+linear\+\_\+acceleration (in mm.\+s-\/2)&0x06&R/W&16 \\\cline{1-4}
max\+\_\+angular\+\_\+acceleration (in °.\+s-\/2)&0x08&R/W&16 \\\cline{1-4}
cruise\+\_\+linear\+\_\+speed(in mm.\+s-\/1)&0x0A&R/W&16 \\\cline{1-4}
cruise\+\_\+angular\+\_\+speed(in °.\+s-\/1)&0x0C&R/W&16 \\\cline{1-4}
linear p coefficient&0x0E&R/W&16 \\\cline{1-4}
linear i coefficient&0x10&R/W&16 \\\cline{1-4}
linear d coefficient&0x12&R/W&16 \\\cline{1-4}
angular p coefficient&0x14&R/W&16 \\\cline{1-4}
angular i coefficient&0x16&R/W&16 \\\cline{1-4}
angular d coefficient&0x18&R/W&16 \\\cline{1-4}
motor left forward sense&0x1A&R/W&8 \\\cline{1-4}
motor right forward sense&0x1B&R/W&8 \\\cline{1-4}
coding wheel left initial ticks low&0x1C&R/W&16 \\\cline{1-4}
coding wheel left initial ticks high&0x1E&R/W&16 \\\cline{1-4}
coding wheel right initial ticks low&0x20&R/W&16 \\\cline{1-4}
coding wheel right initial ticks high&0x22&R/W&16 \\\cline{1-4}
coding wheel left orientation&0x24&R/W&8 \\\cline{1-4}
coding wheel right orientation&0x25&R/W&8 \\\cline{1-4}
store config data in flash&0x30&W&8 \\\cline{1-4}
reserved&0x31-\/0x7F&\\\cline{1-4}
current x absolute position Low&0x80&R/W&16 \\\cline{1-4}
current x absolute position High&0x82&R/W&16 \\\cline{1-4}
current y absolute position Low&0x84&R/W&16 \\\cline{1-4}
current y absolute position High&0x86&R/W&16 \\\cline{1-4}
current right wheel distance Low&0x88&R/W&16 \\\cline{1-4}
current right wheel distance High&0x8A&R/W&16 \\\cline{1-4}
current left wheel distance Low&0x8C&R/W&16 \\\cline{1-4}
current left wheel distance High&0x8E&R/W&16 \\\cline{1-4}
current heading&0x90&R/W&16 \\\cline{1-4}
current mean distance (in mm) Low&0x92&R&16 \\\cline{1-4}
current mean distance (in mm) High&0x94&R&16 \\\cline{1-4}
goal mean distance (in mm) Low&0x\+A0&W&16 \\\cline{1-4}
goal mean distance (in mm) High&0x\+A2&W&16 \\\cline{1-4}
goal heading (in °)&0x\+A4&W&16 \\\cline{1-4}
heading distance sync reference (in mm)&0x\+A6&W&16 \\\cline{1-4}
master stop for motors&0x\+A8&R/W&8 \\\cline{1-4}
\end{longtabu}
